\documentclass{article}
\usepackage{setspace}
%\documentclass[journal]{IEEEtran}
\usepackage[title]{appendix}
\usepackage[dvipsnames]{xcolor}
\usepackage{graphicx}
\usepackage{amsmath}
\usepackage{amssymb}
\usepackage{wrapfig}
\usepackage{mdframed}
\usepackage{amsbsy}
\usepackage{hyperref}
\usepackage{wrapfig}
\usepackage{animate}
\usepackage{algpseudocode}
\usepackage{algorithm}
\usepackage{algcompatible}
\usepackage{tcolorbox}
%\usepackage{caption}
\usepackage[font={small,sf}]{caption}
\usepackage{algorithmicx}
\usepackage{algpseudocode}
\usepackage{float}
\usepackage{multicol}
\usepackage{caption}
\usepackage{amsfonts}
\usepackage{subfig}
\usepackage{color}
\usepackage{subfig}
%\usepackage[demo]{graphicx}
\usepackage{url}
\usepackage[export]{adjustbox}[2011/08/13]
\usepackage{textpos}
\usepackage[percent]{overpic}
%\usepackage{minted}
%\usemintedstyle{vs}
%\usepackage{bbm}


\DeclareMathOperator*{\argmax}{arg\,max}
\DeclareMathOperator*{\argmin}{arg\,min}


%\newcommand{\minus}{\scalebox{0.8}{$-$}}
\newcommand{\conv}{\scalebox{0.6}{$\perp$}}
\newcommand{\como}{\scalebox{0.6}{$+$}}


\definecolor{mygreen}{rgb}{0,0.45,0}
\definecolor{mygray}{rgb}{0.99,0.99,0.99}
\definecolor{mymauve}{rgb}{0.58,0,0.82}


\usepackage{listings,lstautogobble}
\lstset{
	autogobble=true,
	xleftmargin=0.25pt,
	backgroundcolor=\color{mygray},   % choose the background color; you must add \usepackage{color} or \usepackage{xcolor}; should come as last argument
	basicstyle=\small\ttfamily,        % the size of the fonts that are used for the code
	breakatwhitespace=false,         % sets if automatic breaks should only happen at whitespace
	breaklines=true,                 % sets automatic line breaking
	captionpos=t,                    % sets the caption-position to bottom
	commentstyle=\color{mygreen},    % comment style
	%deletekeywords={...},            % if you want to delete keywords from the given language
	escapeinside={\%*}{*)},          % if you want to add LaTeX within your code
	extendedchars=true,              % lets you use non-ASCII characters; for 8-bits encodings only, does not work with UTF-8
	%	frame=single,	                   % adds a frame around the code
	keepspaces=true,                 % keeps spaces in text, useful for keeping indentation of code (possibly needs columns=flexible)
	keywordstyle=\color{blue},       % keyword style
	language=R,                 % the language of the code
	morekeywords={*,...},           % if you want to add more keywords to the set
	numbers=none,                    % where to put the line-numbers; possible values are (none, left, right)
	%	numbersep=2pt,                   % how far the line-numbers are from the code
	numberstyle=\tiny\color{white}, % the style that is used for the line-numbers
	rulecolor=\color{black},         % if not set, the frame-color may be changed on line-breaks within not-black text (e.g. comments (green here))
	showspaces=false,                % show spaces everywhere adding particular underscores; it overrides 'showstringspaces'
	showstringspaces=false,          % underline spaces within strings only
	showtabs=false,                  % show tabs within strings adding particular underscores
	stepnumber=1,                    % the step between two line-numbers. If it's 1, each line will be numbered
	stringstyle=\color{mymauve},     % string literal style
	tabsize=2,	                   % sets default tabsize to 2 spaces
	title=\lstname                   % show the filename of files included with \lstinputlisting; also try caption instead of title
}


\lstdefinestyle{cplusplus}
{
	commentstyle=\color{mygreen}\ttfamily,
	rulecolor=\color{black},
	language=C++
}
\lstnewenvironment{cplusplus}{\lstset{style=cplusplus}}{}


\usepackage{geometry}
\geometry{a4paper, left=1.25in, top=1.25in, right=1.25in, bottom=1.25in}
%\geometry{a4paper, left=1in, top=1in, right=1in, bottom=1in}


%\usepackage[utf8]{inputenc}
%\usepackage{times}

%\renewcommand{\familydefault}{cmr}



\hypersetup{
	colorlinks   = true,
	linkcolor    = blue,
	urlcolor = blue
}

\begin{document}
	
	\title{FJ algorithm correction}
	\author{Charlie Wusuo Liu}
	\date{September 4, 2019}


\section{Rewrite the algorithm}

\begin{algorithm}
	\caption{FJ source code}\label{alg}
	 \textbf{Inputs:} $k_{\min}$, $k_{\max}$, $\epsilon$, initial parameters $\hat{\boldsymbol{\theta}}=\{\hat{\boldsymbol{\theta}}_1,\ldots,\hat{\boldsymbol{\theta}}_{k_{\max}},\hat{\alpha}_1,\ldots,\hat{\alpha}_{k_{\max}}\}$
	 
	\textbf{Output:} Mixture model in $\hat{\boldsymbol{\theta}}_{\text{best}}$

	
	\begin{algorithmic}[1]
		\STATE $t\gets0$, $k_{nz}\gets k_{\max}$, $\mathcal{L}_{\min}\gets+\infty$
     	\STATE $u_{m}^{(i)}\gets p(\boldsymbol{y}^{(i)}|\hat{\boldsymbol{\theta}}_{m})$, for $m=1,\ldots,k_{\max}$, and $i=1,\ldots,n$
     	\WHILE{$k_{nz}\geq k_{\min}$}
     	  \REPEAT
     	  \STATE $t\gets t+1$
     	  \FOR{$m=1$ to $k_{\max}$}
     	  \STATE $w^{(i)}_{m}\gets\hat{\alpha}_mu^{(i)}_m\big(\sum_{j=1}^{k_{\max}}\hat{\alpha}_ju^{(i)}_j\big)^{-1}$, for $i=1,\ldots,n$
     	  \STATE $\hat{\alpha}_m\gets\max\big\{0,\,\big(\sum_{i=1}^{n}w^{(i)}_m\big)-\frac{N}{2}\big\}\,/\,n$
     	  \STATE $\{\hat{\alpha}_1,\ldots,\hat{\alpha}_{k_{\max}}\}\gets\{\hat{\alpha}_1,\ldots,\hat{\alpha}_{k_{\max}}\}\big(\sum_{m=1}^{k_{\max}}\hat{\alpha}_m\big)^{-1}$
     	  \IF{$\hat{\alpha}_m>0$}
     	  \STATE $\hat{\boldsymbol{\theta}}_m\gets\underset{\boldsymbol{\theta}_m}{\argmax}\log p(\mathcal{Y,\mathcal{W}|\boldsymbol{\theta}})$
     	  \STATE $u^{(i)}_m\gets p(\boldsymbol{y}^{(i)}|\hat{\boldsymbol{\theta}}_m)$ for $i=1,\ldots,n$
     	  \ELSE
     	  \STATE $k_{nz}\gets k_{nz}-1$
     	  \ENDIF
     	  \ENDFOR
     	  \STATE $\hat{\boldsymbol{\theta}}(t)\gets\{\hat{\boldsymbol{\theta}}_1,\ldots,\hat{\boldsymbol{\theta}}_{k_{\max}},\hat{\alpha}_1,\ldots,\hat{\alpha}_{k_{\max}}\}$
     	  \STATE $L(t)\gets\sum_{i=1}^{n}\log\sum_{m=1}^{k}\hat{\alpha}_mu^{(i)}_m$
     	  \UNTIL{$|L(t)-L(t-1)|<\epsilon|L(t-1)|$}
     	  \STATE $\mathcal{L}[\hat{\boldsymbol{\theta}}(t),\mathcal{Y}]\gets\frac{N}{2}\underset{m:\hat{\alpha}_m>0}{\sum}\log\hat{\alpha}_m+\frac{1}{2}k(N+1)\log n-L(t)$
     	 \IF{$\mathcal{L}[\hat{\boldsymbol{\theta}}(t),\mathcal{Y}]\leq\mathcal{L}_{\min}$}
     	 \STATE $\mathcal{L}_{\min}\gets\mathcal{L}[\hat{\boldsymbol{\theta}}(t),\mathcal{Y}]$
     	 \STATE $\hat{\boldsymbol{\theta}}_{\text{best}}\gets\hat{\boldsymbol{\theta}}(t)$
     	 \ENDIF
     	\STATE $m^*\gets\underset{m}{\argmin}\{\hat{\alpha}_m>0\}$, $\hat{\alpha}_{m^*}\gets0$, $k_{nz}\gets k_{nz}-1$
     	\STATE $\{\hat{\alpha}_1,\ldots,\hat{\alpha}_{k_{\max}}\}\gets\{\hat{\alpha}_1,\ldots,\hat{\alpha}_{k_{\max}}\}\big(\sum_{m=1}^{k_{\max}}\hat{\alpha}_m\big)^{-1}$
		\ENDWHILE
	\end{algorithmic}
\end{algorithm}





%\begin{algorithm}
%	\caption{FJ GMM source code}\label{sbs}
%%	\textbf{LEFT BRANCH}:
%	\begin{algorithmic}[1]
%		\STATE $\beta\gets 0$\,. $\triangleleft$ $\beta=0$ implies the left branch, 1 the right branch.
%		\STATE Copy $n$, dimensionality of the parent hyperrectangle;
%		\STATE Copy MIN, MAX, $\sum\limits_{t=0}^{n-1}\boldsymbol{x}\big(l(i_t)\big)$ , $\sum\limits_{t=0}^{n-1}\boldsymbol{x}\big(u(i_t)\big)$ , $l$ and $u$ from the parent.
%		\STATE\label{contract}Update the copied parameters through contraction. If it fails, \textbf{return} a failure signal.
%		\STATE\label{findt}$T\gets\{t|u(i_t)=l(i_t)\}$ , $n_z\gets|T|$ . $\triangleleft$ $|T|$ is the cardinality.
%		\STATE $l\gets\{l(i_t)|t\notin T\}$ , $u\gets\{u(i_t)|t\notin T\}$ , $n\gets n-n_z$ .
%		\STATE Update $\sum\limits_{t=0}^{n-1}\boldsymbol{x}\big(l(i_t)\big)$ and $\sum\limits_{t=0}^{n-1}\boldsymbol{x}\big(u(i_t)\big)$ .
%		\STATE\label{push}Push $\{u(i_t)|t\in T\}$ in a global buffer $B$ that is to hold a qualified subset. $\triangleleft$ This step goes concurrently with Step \ref{findt} .
%		\STATE $\kappa\gets\underset{t}{\argmin}\big(u(i_t)-l(i_t)\big)$ .
%		\STATE $u^{\prime}\gets\{u(i_0),\ldots,u(i_\kappa)\}$ , $S(u)\gets\sum\limits_{t=0}^{n-1}\boldsymbol{x}\big(u(i_t)\big)$ .
%		\STATE For $t\in[0,\,\kappa]$ , $u(i_t)\gets\min\big(u(i_t),\,\lfloor u(i_\kappa)/2\rfloor-\kappa+t\big)$ . $\triangleleft$ Loop $t$ from $\kappa$ and stop once $u(i_t)\leq\lfloor u(i_\kappa)/2\rfloor-\kappa+t$ .
%		\STATE Update $\sum\limits_{t=0}^{n-1}\boldsymbol{x}\big(l(i_t)\big)$ , $\sum\limits_{t=0}^{n-1}\boldsymbol{x}\big(u(i_t)\big)$ . $\triangleleft$ Use $\mathcal{M}$ for fast update. 
%	\end{algorithmic}
%%	If contraction succeeds in Step \ref{contract}, move to the right hyperrectangle in stack and execute the above steps again. Otherwise left-propagate through stack while erasing the last $n_z$ elements in buffer $B$ for each hyperrectangle, and stop once the current one has $\beta=0$ .
%%	
%%	\textbf{RIGHT BRANCH}:
%%	\begin{algorithmic}[1]
%%		\STATE $\beta\gets 1$ .
%%		\STATE For $t\in[0,\,\kappa]$ , $u(i_t)\gets u^{\prime}(i_t)$ ; $\sum\limits_{t=0}^{n-1}\boldsymbol{x}\big(u(i_t)\big)\gets S(u)$ .
%%		\STATE For $t\in[\kappa,\,n-1]$ , $l(i_t)\gets\max\big(l(i_t),\,u(i_\kappa)+1+t-\kappa\big)$ . $\triangleleft$ Loop $t$ from $\kappa$ and stop once $l(i_t)\geq u(i_\kappa)+1+t-\kappa$ .
%%		\STATE Update $\sum\limits_{t=0}^{n-1}\boldsymbol{x}\big(l(i_t)\big)$ . $\triangleleft$ Use $\mathcal{M}$ for fast update.
%%	\end{algorithmic}
%%	Move to the right hyperrectangle and execute \textbf{LEFT BRANCH}.
%\end{algorithm}




\end{document}














